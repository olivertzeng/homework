\documentclass{article}
\usepackage{xeCJK}
\setCJKmainfont{FakePearl}
\usepackage{hyperref}

\usepackage[dvipsnames]{xcolor}
\usepackage{tikz}
\usetikzlibrary{ shapes.geometric }
\usetikzlibrary{calc}
\usepackage{anyfontsize}


\begin{document}
\pagestyle{empty}

\begin{tikzpicture}[remember picture,overlay]
%%%%%%%%%%%%%%%%%%%% Background %%%%%%%%%%%%%%%%%%%%%%%%
\fill[ForestGreen] (current page.south west) rectangle (current page.north east);

\foreach \i in {2.5,...,22}
{
    \node[rounded corners,ForestGreen!60,draw,regular polygon,regular polygon sides=6, minimum size=\i cm,ultra thick] at ($(current page.west)+(2.5,-5)$) {} ;
}

%%%%%%%%%%%%%%%%%%%% Background Polygon %%%%%%%%%%%%%%%%%%%%
\foreach \i in {0.5,...,22}
{
\node[rounded corners,ForestGreen!60,draw,regular polygon,regular polygon sides=6, minimum size=\i cm,ultra thick] at ($(current page.north west)+(2.5,0)$) {} ;
}

\foreach \i in {0.5,...,22}
{
\node[rounded corners,ForestGreen!90,draw,regular polygon,regular polygon sides=6, minimum size=\i cm,ultra thick] at ($(current page.north east)+(0,-9.5)$) {} ;
}


\foreach \i in {21,...,6}
{
\node[ForestGreen!85,rounded corners,draw,regular polygon,regular polygon sides=6, minimum size=\i cm,ultra thick] at ($(current page.south east)+(-0.2,-0.45)$) {} ;
}


%%%%%%%%%%%%%%%%%%%% Title of the Report %%%%%%%%%%%%%%%%%%%%
\node[left,ForestGreen!5,minimum width=0.625*\paperwidth,minimum height=3cm, rounded corners] at ($(current page.north east)+(0,-9.5)$)
{
{\fontsize{25}{30} \selectfont \bfseries 個人作業 3}
};

%%%%%%%%%%%%%%%%%%%% Subtitle %%%%%%%%%%%%%%%%%%%%
\node[left,ForestGreen!10,minimum width=0.625*\paperwidth,minimum height=2cm, rounded corners] at ($(current page.north east)+(0,-11)$)
{
{\huge \textit{演化概念發展}}
};

%%%%%%%%%%%%%%%%%%%% Author Name %%%%%%%%%%%%%%%%%%%%
\node[left,ForestGreen!5,minimum width=0.625*\paperwidth,minimum height=2cm, rounded corners] at ($(current page.north east)+(0,-13)$)
{
{\Large \textsc{曾嘉禾}}
};

%%%%%%%%%%%%%%%%%%%% Year %%%%%%%%%%%%%%%%%%%%
\node[rounded corners,fill=ForestGreen!70,text =ForestGreen!5,regular polygon,regular polygon sides=6, minimum size=2.5 cm,inner sep=0,ultra thick] at ($(current page.west)+(2.5,-5)$) {\LARGE \bfseries 2024};

\end{tikzpicture}
\newpage
\begin{frame}
真菌常常是人類感到麻煩的生物,常常東西一受潮就會遭到他們的襲擊。但是近年來科學家發現某些真菌可以利用他們的「軟實力」完成一些複雜的問題。甚至發展出「\textbf{真菌電腦}」一詞。光是低等的真菌—粘菌都可以解決迷宮問題或是極為複雜,連現階段的電腦都無法快速計算的旅行推銷員問題(演算法 Big O(n!))。真菌之所以能夠傳遞資料是因為其如同大腦中的神經細胞的功能一般,會傳遞微弱的電流以傳遞與接收資訊。科學家甚至找到能夠利用其實力寫出程式來請他「執行」就如同我們的腦袋一般。這使得人類無法想像那平常在麵包上黑黑的東西其實這麼厲害呢!不過即使真菌有過人的計算能力,他們仍然有些限制需要有所突破。例如在「翻譯」電流開關到可讀的二位元資訊的技術目前還不足,乾冷等環境更是他們的限制。不過若人類想要使用它當作電腦更是要解決最關鍵的限制—病菌感染。若使用者未有常常備份資料的習慣,若真菌不幸大量死亡,那資料可能就不負存。

即使有這些限制,但相信未來的科學家能將「真」正的「軟」實力開發到極致,普及於大眾!

Source: \href{https://www.youtube.com/watch?v=5mIWo6dgTmI}{The Mushroom Motherboard: The Crazy Fungal Computers that Might Change Everything}
\end{frame}

\end{document}
