\documentclass{article}
\usepackage{settings}

\begin{document}
\pagestyle{empty}

\begin{tikzpicture}[remember picture,overlay]
%%%%%%%%%%%%%%%%%%%% Background %%%%%%%%%%%%%%%%%%%%%%%%
\fill[ForestGreen] (current page.south west) rectangle (current page.north east);

\foreach \i in {2.5,...,22}
{
    \node[rounded corners,ForestGreen!60,draw,regular polygon,regular polygon sides=6, minimum size=\i cm,ultra thick] at ($(current page.west)+(2.5,-5)$) {} ;
}

%%%%%%%%%%%%%%%%%%%% Background Polygon %%%%%%%%%%%%%%%%%%%%
\foreach \i in {0.5,...,22}
{
\node[rounded corners,ForestGreen!60,draw,regular polygon,regular polygon sides=6, minimum size=\i cm,ultra thick] at ($(current page.north west)+(2.5,0)$) {} ;
}

\foreach \i in {0.5,...,22}
{
\node[rounded corners,ForestGreen!90,draw,regular polygon,regular polygon sides=6, minimum size=\i cm,ultra thick] at ($(current page.north east)+(0,-9.5)$) {} ;
}


\foreach \i in {21,...,6}
{
\node[ForestGreen!85,rounded corners,draw,regular polygon,regular polygon sides=6, minimum size=\i cm,ultra thick] at ($(current page.south east)+(-0.2,-0.45)$) {} ;
}


%%%%%%%%%%%%%%%%%%%% Title of the Report %%%%%%%%%%%%%%%%%%%%
\node[left,ForestGreen!5,minimum width=0.625*\paperwidth,minimum height=3cm, rounded corners] at ($(current page.north east)+(0,-9.5)$)
{
{\fontsize{25}{30} \selectfont \bfseries 個人作業 3}
};

%%%%%%%%%%%%%%%%%%%% Subtitle %%%%%%%%%%%%%%%%%%%%
\node[left,ForestGreen!10,minimum width=0.625*\paperwidth,minimum height=2cm, rounded corners] at ($(current page.north east)+(0,-11)$)
{
{\huge \textit{演化概念發展}}
};

%%%%%%%%%%%%%%%%%%%% Author Name %%%%%%%%%%%%%%%%%%%%
\node[left,ForestGreen!5,minimum width=0.625*\paperwidth,minimum height=2cm, rounded corners] at ($(current page.north east)+(0,-13)$)
{
{\Large \textsc{曾嘉禾}}
};

%%%%%%%%%%%%%%%%%%%% Year %%%%%%%%%%%%%%%%%%%%
\node[rounded corners,fill=ForestGreen!70,text =ForestGreen!5,regular polygon,regular polygon sides=6, minimum size=2.5 cm,inner sep=0,ultra thick] at ($(current page.west)+(2.5,-5)$) {\LARGE \bfseries 2024};

\end{tikzpicture}
\newpage
\begin{boxpar}{內文}
在生物學家對生物的演化史感到好奇前,人們為了要將生物分類,林奈在 1758 年創造了二名法,將每種生物以系統化的方式命名。這個發明讓後人更容易研究生物的特徵以及其演化史之間的關聯。

「生物多樣性到底是怎麼形成的?為什麼世上有如此多種不同生物?」等問題是人們一直以來想解開的謎團。林奈的二名法雖然未直接揭開上述的謎底,但他的貢獻為後續的生物學研究奠定了基礎。

在同世紀內,有兩位好奇生物的由來的生物學家也同時著手於此。他們認為生物的品種會隨時間變化。其中,馬爾薩斯在 1798 年所提出的馬爾薩斯模型指出未來人口將以指數型成長,若在糧食生產等比成長下將發生糧食危機。雖然此模型仍未考慮環境負載力,馬爾薩斯災難—也就是大饑荒並未實際發生。然而,他的理論仍然有助於生物史研究的發展,因為生物的種類的確在最初是呈指數型成長的。

拉馬克於 1801 年時提出複雜的生物是由簡化繁的想法,不過未立即得到認同。

同時英國科學家威廉史密斯對於地殼內部的研究發現化石越靠近上層越年輕,反之越底層的年代越久遠。這個發現為生物演化史的研究提供了重要的證據。

在 1830 年時萊爾的均變論主張地質事件是持續性的,但同時也可能會如同維爾納與居維葉的災辯論一般發生突發事件,如隕石。雖然如此,萊爾理論還是否定了達爾文的天擇說。

最後達爾文在 1859
    年提出的演化論與天擇說指出生物的種類數量呈指數成長,故大自然會淘汰某些種類以維持環境負載。甚至有可能會將某種族群滅絕。\\
Source: \href{https://www.youtube.com/watch?v=5mIWo6dgTmI}{The Mushroom Motherboard: The Crazy Fungal Computers that Might Change Everything}
\end{boxpar}

\end{document}
