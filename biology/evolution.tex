\documentclass[
	a4paper,
	fontsize=11pt,
	twoside=true,
	numbers=noenddot,
]{article}

\usepackage{xeCJK}
\usepackage{enumerate}
\usepackage[top=1.5cm, bottom=1.5cm, left=3cm, right=3cm]{geometry}
\usepackage{hyperref}
\setCJKmainfont{Noto Serif CJK TC}

\title{\textbf{古代地球,原始生命:布豐\\}
\LARGE Old Earth, Ancient Life:\\
\LARGE Georges-Louis Leclrec, Comte de Buffon\\
\LARGE Made With \LaTeX
}
\date{\today}
\author{作者:曾嘉禾、潘彥儒、林學佑}
\begin{document}
\maketitle

\section{原文大意}
Buffon was both an innovative Enlightenment thinker, and a brave
scientist. Not only he contributed to the encyclopedia which amassed an
overwhelming amount of knowledge, he was also the steppingstone of
biology, who explained how migrations affected organisms to form
different ethnicities and communities, accelerated evolutions, and
inherited their traits to their offsprings, giving birth to distinctive
local species till today. He also estimated that the Earth is much older
than what people in the 1600s thought. Nevertheless, his dearth of
evidence and underestimation of the age of the Earth resulted in his
hypotheses. Despite the failure, his inventive thoughts influenced
beyond the whole biology scope, including the ever-famous Darwin who
continued on what he had left, fixed what he missed, and managed to
succeed soon after.

\section{中文翻譯大意}
布豐是在1700年代的科學家以及啟蒙家。他當時不但蒐錄了包羅萬象的知識於百科全書內,還傳承了啟蒙者理性的精神,以科學的角度推測出生物的發展史,以及地球的起源。他認為地球早在七萬年就誕生了,比上世紀的人們的猜測遠高出十倍。後來更是提出生物大遷徙與演化的關聯。群集以及族群的形成,不是先天而是後天的結果。然而,即使他的假說不但欠缺證據,更是低估了地球的年紀,不過他對於科學界的影響的確功不可沒。他的貢獻,更是啟發百年後的著名生物學家─達爾文的墊腳石。最後在他們兩個的努力下,生物學又邁進了一大步。

\section{多選題}
布豐伯爵的科學貢獻,下列哪些陳述是正確的?\\

\begin{enumerate}[(A)]
    \item 布豐將包羅萬象的知識收錄於一部百科全書中,並傳承了啟蒙者理性的精神。
    \item 布豐認為地球大約誕生於七萬年前,這比當時人們的猜測高出十倍。
    \item 布豐提出了生物大遷徙與演化的關聯,認為群集及族群的形成是後天的結果。
    \item 布豐的假說完全符合當時的科學證據並得到普遍認可。
    \item 布豐的工作對科學界影響深遠,啟發了百年後達爾文的理論發展。
\end{enumerate}
答案:(A)、(B)、(C)、(E)\\

(D)錯誤原因:布豐的假說在當時缺乏充分的證據支持,且低估了地球的年齡,因此未能完全符合當時的科學證據。他的理論雖然有遠見,但因證據不足而未能得到普遍認可。\\

來源: \href{https://evolution.berkeley.edu}{Understanding Evolution}
\end{document}
