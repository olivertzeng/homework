\documentclass[11pt]{article}
\usepackage{xeCJK}
\setCJKmainfont{Noto Serif CJK TC}
\usepackage{hyperref}

\begin{document}
\begin{frame}
Buffon was both an innovative Enlightenment thinker, and a brave
scientist. Not only he contributed to the encyclopedia which amassed an
overwhelming amount of knowledge, he was also the steppingstone of
biology, who explained how migrations affected organisms to form
different ethnicities and communities, accelerated evolutions, and
inherited their traits to their offsprings, giving birth to distinctive
local species till today. He also estimated that the Earth is much older
than what people in the 1600s thought. Nevertheless, his dearth of
evidence and underestimation of the age of the Earth resulted in his
hypotheses. Despite the failure, his inventive thoughts influenced
beyond the whole biology scope, including the ever-famous Darwin who
continued on what he had left, fixed what he missed, and managed to
succeed soon after.

布豐是在1700年代的科學家以及啟蒙家。他當時不但蒐錄了包羅萬象的知識於百科全書內,還傳承了啟蒙者理性的精神,以科學的角度推測出生物的發展史,以及地球的起源。他認為地球早在七萬年就誕生了,比上世紀的人們的猜測遠高出十倍。後來更是提出生物大遷徙與演化的關聯。群集以及族群的形成,不是先天而是後天的結果。然而,即使他的假說不但欠缺證據,更是低估了地球的年紀,不過他對於科學界的影響的確功不可沒。他的貢獻,更是啟發百年後的著名生物學家─達爾文的墊腳石。最後在他們兩個的努力下,生物學又邁進了一大步。

Source: \href{https://evolution.berkeley.edu}{Understanding Evolution}
\end{frame}
\end{document}
