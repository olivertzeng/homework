\documentclass{article}
\usepackage{../../settings}

\begin{document}
\hexcover{P3 科學筆記本}{學習歷程}{作者:曾嘉禾}{Orchid}{16260621}

\begin{large}
\begin{boxpar}{Orchid}{學習歷程檔案簡述}
    此次實驗討論了催化劑與反應速率的關係,根據不同的環境以及加入不同物質討論不同操作變因對於反應速率的影響。在此次小組實驗中,我雖然是報告者,但為了實驗的完整性也常常參與組員之間的討論,希望可以做出最周全的實驗。除了在講台上發揮外,我也在製作海報時提供了大量建議。即使最後因為時間限制無法逾期完成海報,有望可以再次改善組內的工作效率。
\end{boxpar}
\begin{boxpar}{Orchid}{學習歷程需要有的}
    \begin{itemize}
\item 動機
\item 學習歷程內容
\item 實驗過程
\item 學習歷程心得
    \end {itemize}
\end{boxpar}
\begin{boxpar}{Orchid}{實驗步驟}
    \begin{enumerate}
\item 發現問題
\item 提出假說
\item 規劃設計實驗
\item 實驗過程
\item 驗證結果與假說
    \end {enumerate}
\end{boxpar}
\end{large}
\end{document}
