\documentclass{article}
\usepackage{../../settings}

\begin{document}
\hexcover{Py 心寫程式}{學習歷程}{作者:曾嘉禾}{Dandelion}{電算社}

\begin{large}
\begin{boxpar}{Dandelion}{說明}
利用 Python 寫出像樣的遊戲
\end{boxpar}
\begin{boxpar}{Dandelion}{動機}
雖然我有在寫程式,但常常好奇在那個黑盒子外還可以用程式寫出其他很酷的應用,很經典的例子就是遊戲。一直以來,我更是好奇程式是如何處理視窗、按鈕、元件等等的圖形化介面的項目。來到師大附中博覽會,看到電算社能夠利用python
的 pygame
做成各式各樣好玩遊戲的樣子尤其感到羨慕。所以這次為了圓夢以及完成擔任電算社靜態展的義務,決定與組員以
pygame 寫出像樣的遊戲,花點心思懷抱與追求自己的夢想。
\end{boxpar}
    \begin{boxpar}{Dandelion}{學習過程}
        在學習 pygame 的剛開始,接觸到的最基本的 pygame 程式碼後驚訝的發現竟然有 while true:
        的危險程式,但後來才知道只要讓程式碼有條件地退出就不會產生太多問題
        \begin{mintbox}{範例程式碼}{Dandelion}{python}
# Example file showing a basic pygame "game loop"
import pygame

# pygame setup
pygame.init()
screen = pygame.display.set_mode((1280, 720))
clock = pygame.time.Clock()
running = True

while running:
    # poll for events
    # pygame.QUIT event means the user clicked X to close your window
    for event in pygame.event.get():
        if event.type == pygame.QUIT:
            running = False

    # fill the screen with a color to wipe away anything from last frame
    screen.fill("purple")

    # RENDER YOUR GAME HERE

    # flip() the display to put your work on screen
    pygame.display.flip()

    clock.tick(60)  # limits FPS to 60

pygame.quit()
        \end{mintbox}
    \end{boxpar}
\begin{boxpar}{遇到的問題}{Dandelion}

\end{boxpar}
\end{large}
\end{document}
