\documentclass{article}
\usepackage{../../settings}

\begin{document}
\hexcover{APCS 從零到三級跳}{學習歷程}{作者:曾嘉禾}{Tan}{2025}

\begin{large}
\begin{boxpar}{Tan}{就讀動機}
\end{boxpar}
通過APCS實作三級考試後,儘管我的程式設計能力已超越大多數高中生,但我認知到資訊工程不僅僅局限於程式設計,還涵蓋了諸如資訊安全、遊戲開發、行動裝置、網路技術、系統工程、積體電路設計、人工智慧等廣泛的領域知識。我希望能夠有機會深入探索這些領域,並專注於自己感興趣的領域,做出有意義的貢獻。
\begin{boxpar}{Tan}{成績證明}
    \includegraphics[width=\linewidth]{src/proof.png}
\end{boxpar}
\begin{boxpar}{Tan}{反思}
從零開始學程式,到考到APCS實作三級中,我不斷地坐在電腦前,絞盡腦汁地找出自己腦中的邏輯,並且運用所學到的演算法解題,不但帶來成就感,也讓我實際地在最後的
APCS 測驗中考到不錯的成績
\end{boxpar}
\begin{boxpar}{Tan}{未來規劃}
以前在國中二年級的時候因緣際會地認識 Reddit、iOS
    越獄、Linux、與開放原始碼自由的世界(\url{https://github.com/olivertzeng}),一年後甚至自己靠著 Arch
Wiki 的力量成功安裝 Arch
    Linux。在這個過程中,我認知到了開放原始碼的重要以及其對於世界的貢獻。例如以台灣人嚴育銓在瑞士所創立的質子公司為例,它們利用開源的力量使許多人能夠安全且簡易地以
    PGP 加密所有送出的電子郵件以對抗 Gmail
    在市面上的壟斷。所以在未來我將利用寫成是的力量寫出更多開源程式碼,讓更多人能夠更安心的使用我的程式甚至衍申出更多新創意。
\end{boxpar}
\end{large}
\end{document}
