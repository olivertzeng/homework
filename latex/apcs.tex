\documentclass{article}
\usepackage{../settings}

\begin{document}
\renewcommand{\baselinestretch}{1.5}
\hexcover{APCS 題本}{範例觀念題}{作者:廖家緯、曾嘉禾、APCS 出題}{SkyBlue}{電算社 2024}


\begin{sblock}{SkyBlue}

2. 給定右側函式f(),當執行f(10)時,最終回傳結果為何?\\

\begin{enumerate}[(A)]
    \item 1
    \item 3840
    \item -3840
    \item 執行時導致無限迴圈,不會停止執行
\end{enumerate}

\tcblower
\begin{minted}[fontsize=\small]{c}
int f(int i) {
	if (i > 0)
		if ((i / 2) % 2 == 0)
			return f(i - 2) * i;
		else
			return f(i - 2) * (-i);
	else
		return 1;
}
\end{minted}
\end{sblock}

\begin{sblock}{SkyBlue}

31. 右側 g(4)函式呼叫執行後,回傳值為何?\\

\begin{enumerate}[(A)]
    \item 6
    \item 11
    \item 13
    \item 14
\end{enumerate}

\tcblower

\begin{minted}[fontsize=\small]{c}
int f(int n) {
	if (n > 3) {
		return 1;
	} else if (n == 2) {
		return (3 + f(n + 1));
	} else {
		return (1 + f(n + 1));
	}
}
int g(int n) {
	int j = 0;
	for (int i = 1; i <= n - 1; i = i + 1) {
		j = j + f(i);
	}
	return j;
}
\end{minted}

\end{sblock}
\begin{sblock}{SkyBlue}

32. 右側 Mystery()函式 else 部分運算式應為何,才能使得 Mystery(9) 的回傳值為 34。\\

\begin{enumerate}[(A)]
	\item x + Mystery(x - 1)
	\item x * Mystery(x - 1)
	\item Mystery(x - 2) + Mystery(x + 2)
	\item Mystery(x - 2) + Mystery(x - 1)
\end{enumerate}

\tcblower
\begin{minted}[fontsize=\small]{c}
int Mystery(int x) {
	if (x <= 1) {
		return x;
	} else {
		return ____________;
	}
}
\end{minted}
\end{sblock}

來源:
\href{https://apcs.csie.ntnu.edu.tw/wp-content/uploads/2022/10/%E8%A7%80%E5%BF%B5%E9%A1%8C_%E9%A1%8C%E5%9E%8B%E7%AF%84%E4%BE%8B.pdf}{原始題目連結}
\end{document}
