\documentclass{article}
\usepackage{../settings}

\begin{document}
\hexcover{討論議題}{主題九}{作者:曾嘉禾、廖瑞鴻、賴冠名}{Cyan}{第三組}

\begin{large}
\tcbset{enhanced,fonttitle=\bfseries\large,fontupper=\normalsize\sffamily,
colback=Cyan!10!white,colframe=Cyan!50!black,colbacktitle=Cyan!30!white,
coltitle=black,center title}

\begin{document}

\begin{tcolorbox}[tabularx={c||l|l|l|l|l},title=取材自錢宜新、姚珩]
& 水星   & 金星   & 火星   & 木星    & 土星\\\hline\hline
    兩次衝(或內合)時間間隔(\textit{t})       & 0.32 & 1.60 & 2.19 & 1.09  & 1.04\\\hline
衝至方照的時間間隔($\tau$)          &      &      & 0.27 & 0.25  & 0.24\\\hline
內合至大具的時間間隔($\tau$)         & 0.06 & 0.19 &      &       &  \\\hline
    公轉的週期理論值              & 0.24 & 0.62 & 1.84 & 12.19 & 27.04 \\\hline
公認值                   & 0.24 & 0.62 & 1.88 & 11.86 & 29.46 \\\hline
軌道半徑理論值               & 0.38 & 0.72 & 1.43 & 6.59  & 8.95  \\\hline
公認值                   & 0.28 & 0.72 & 1.52 & 5.20  & 9.55
\end{tcolorbox}

\captionof{透過觀察數據\textit{t}與$\tau$,及經由簡單的日新說模型,所求得的公轉週期(單位:年)與軌道半徑(單位:地球公轉半徑)之理論值}

\label{tab:my-table}
\newcolumntype{Y}{>{\raggedleft\arraybackslash}X}% see tabularx
\tcbset{enhanced,fonttitle=\bfseries\large,fontupper=\normalsize\sffamily,
colback=yellow!10!white,colframe=red!50!black,colbacktitle=Salmon!30!white,
coltitle=black,center title}
\begin{tcolorbox}[tabularx={X||Y|Y|Y|Y||Y},title=My table]
Group & One & Two & Three & Four & Sum\\\hline\hline
Red & 1000.00 & 2000.00 & 3000.00 & 4000.00 & 10000.00\\\hline
Green & 2000.00 & 3000.00 & 4000.00 & 5000.00 & 14000.00\\\hline
Blue & 3000.00 & 4000.00 & 5000.00 & 6000.00 & 18000.00\\\hline\hline
Sum & 6000.00 & 9000.00 & 12000.00 & 15000.00 & 42000.00
\end{tcolorbox}
\end{large}
\end{document}
