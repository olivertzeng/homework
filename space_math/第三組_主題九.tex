\documentclass{article}
\usepackage{../settings}

\begin{document}
\hexcover{討論議題}{主題九}{作者:曾嘉禾、廖瑞鴻、賴冠名}{Cyan}{第三組}

\begin{large}
\tcbset{enhanced,fonttitle=\bfseries\large,fontupper=\normalsize\sffamily,
colback=Cyan!10!white,colframe=Cyan!50!black,colbacktitle=Cyan!30!white,
coltitle=black,center title}

\begin{document}


\begin{enumerate}
    \item 行星的英文 planet 源自希臘文之 aster planetes,意思是wandering star, 為什麼?
    \begin{textbox}{答案}{Cyan}
		由於其他行星對於地球而言有時會朝平時公轉的反方向前進(俗稱逆行)過一陣子又會回到同一個方向。故人們稱之為
                wandering star,漫步的星星。
    \end{textbox}
    \item 討論表一 \ref{tab:tab1}。
    \item
        如果地球繞太陽運行,則哥白尼可推出地球自轉,太陽和群星東昇西落,都是地球自轉的效應,如何推論?
    \item
        托勒密這一套計算法看來並沒有被哥白尼打敗(因為哥白尼自己也用本輪軍輪,只不過把地心換成日心)托勒密這一套究竟撐到什麼時候才被大多數的天文學家放棄?
        \begin{enumerate}[a.]
            \item 黑金剛大哥什麼時候消失?
            \item 條碼機什麼時候席捲所有超市?
            \item 貨幣的金本位制什麼時候被放棄?
            \item 九九乘法表什麼時候不再背了?
            \item 熱菜什麼時候開始用微波爐?
            \item 紙本對數表什麼時候被網路搜尋取代?
            \item 什麼時候不用再學數學了?
        \end{enumerate}
    \item
        伽利略因宣揚日新說而遭教廷判終身軟禁,目前教廷態度如何?(網路搜尋:維基百科,伽利略,天主教對伽利略的重新認定)
    \item 許多科學史家認為 1543 年哥白尼出版《天體運行論》是科學革命的開始,為什麼?
    \item 文藝復興、科學革命和啟蒙運動的關聯如何?
    \item 依你之見,中國歷史上曾經有過文藝復興、科學革命會啟蒙運動嗎?
\end{enumerate}

\begin{tcolorbox}[tabularx={c||l|l|l|l|l},title=取材自錢宜新、姚珩]
& 水星   & 金星   & 火星   & 木星    & 土星\\\hline\hline
    兩次衝(或內合)時間間隔(\textit{t})       & 0.32 & 1.60 & 2.19 & 1.09  & 1.04\\\hline
衝至方照的時間間隔($\tau$)          &      &      & 0.27 & 0.25  & 0.24\\\hline
內合至大具的時間間隔($\tau$)         & 0.06 & 0.19 &      &       &  \\\hline
    公轉的週期理論值              & 0.24 & 0.62 & 1.84 & 12.19 & 27.04 \\\hline
公認值                   & 0.24 & 0.62 & 1.88 & 11.86 & 29.46 \\\hline
軌道半徑理論值               & 0.38 & 0.72 & 1.43 & 6.59  & 8.95  \\\hline
公認值                   & 0.28 & 0.72 & 1.52 & 5.20  & 9.55
\end{tcolorbox}

\captionof{表一:透過觀察數據\textit{t}與$\tau$,及經由簡單的日新說模型,所求得的公轉週期(單位:年)與軌道半徑(單位:地球公轉半徑)之理論值}
\label{tab:tab1}

\end{large}
\end{document}
